% This file includes many commands that I use frequently when creating TeX documents.


%%%%%%%%%%%%%%%%%%%%%%%%%%%%%%%%%%%%%%%%%%%%%%%%%%%%%%%%
% Packages (alphabetical order)
%%%%%%%%%%%%%%%%%%%%%%%%%%%%%%%%%%%%%%%%%%%%%%%%%%%%%%%%

\usepackage{amsfonts}
\usepackage{amsmath}
\usepackage{amssymb}
\usepackage{amstext}
\usepackage{amsthm}
\usepackage{booktabs}
\usepackage[tableposition=top]{caption}
\usepackage{color}
\usepackage{colortbl}
\usepackage{enumerate}
\usepackage{float}    \floatstyle{plaintop} \restylefloat{table} % Make captions on top of a table (where it should be)
\usepackage{graphicx}
\usepackage{hyperref}
\usepackage{latexsym}
\usepackage{listings}
\usepackage{marvosym}
\usepackage{multirow}
\usepackage{pslatex}
\usepackage{subfigure}
\usepackage{tikz} \usetikzlibrary{matrix,decorations.pathreplacing}
\usepackage{url}
\usepackage{xspace}

%%%%%%%%%%%%%%%%%%%%%%%%%%%%%%%%%%%%%%%%%%%%%%%%%%%%%%%%
% Useful commands
%%%%%%%%%%%%%%%%%%%%%%%%%%%%%%%%%%%%%%%%%%%%%%%%%%%%%%%%

\newcommand{\etal}{\emph{et al.}\xspace}
\newcommand{\CITE}{{\bf {\color{red} [CITE]}}\xspace}
\newcommand{\REF}{{\bf {\color{red} [REF]}}\xspace}
\newcommand{\FIGURE}{{\bf {\color{red} [FIG]}}}
\newcommand{\NOTE}[1]{\renewcommand{\fboxsep}{1mm}\framebox{\textsf{\em #1}}\marginpar{\bf$\leftarrow$NOTE}}
\newcommand{\CONTENT}{{\medskip \center{\Huge \ldots} \marginpar{\bf$\leftarrow$MORE} \medskip}}
\newcommand{\bigO}[1]{\mathcal{O}(#1)\xspace}
\newcommand{\littleo}[1]{o(#1)\xspace}
\newcommand{\Gram}{\text{Gram}}
\newcommand{\tr}{\text{Tr}}
\newcommand{\aip}[2]{| \langle #1 , #2 \rangle |}
\newcommand{\BR}{\{e^{\frac{2\pi i}{3}},e^{-\frac{2\pi i}{3}}\}}
\renewcommand{\bar}[1]{\overline{#1}}    % Overline looks nicer for conjugation...
\newcommand{\con}{\rightarrow\leftarrow} % Contradiction

% Nicer looking gray
\definecolor{Gray}{gray}{0.9}

%%%%%%%%%%%%%%%%%%%%%%%%%%%%%%%%%%%%%%%%%%%%%%%%%%%%%%%%
% Theorems / Lemmas / Remarks / etc.
%%%%%%%%%%%%%%%%%%%%%%%%%%%%%%%%%%%%%%%%%%%%%%%%%%%%%%%%
\theoremstyle{plain}
\newtheorem{theorem}{Theorem}[chapter]
\newtheorem{lemma}[theorem]{Lemma}
\newtheorem{corollary}[theorem]{Corollary}
\newtheorem{proposition}[theorem]{Proposition}

\theoremstyle{definition}
\newtheorem{definition}[theorem]{Definition}
\newtheorem{conjecture}[theorem]{Conjecture}
\newtheorem{example}[theorem]{Example}

\theoremstyle{remark}
\newtheorem{remark}[theorem]{Remark}
\newtheorem{notation}[theorem]{Notation}
\newtheorem{note}[theorem]{Note}
\newtheorem{summary}[theorem]{Summary}
\newtheorem{problem}[theorem]{Problem}
\newtheorem{construction}[theorem]{Construction}

% I like using arrays to hold equations (lines up the '=' nicely)
% They need to be vertically spaced different than a normal matrix, though
% When you make an array for equation, do something like this:
%   \equationarray
%   \begin{array}{lcr}
%     f(x) & = & (x-2)^2 \\
%          & = & x^2 - 4x + 4
%   \end{array}
%   \normalarray
\newcommand{\equationarray}{\renewcommand{\arraystretch}{1.55}}
\newcommand{\normalarray}{\renewcommand{\arraystretch}{1.1}}
\newcommand{\appendixarray}{\renewcommand{\arraystretch}{1}}

\normalarray

%%%%%%%%%%%%%%%%%%%%%%%%%%%%%%%%%%%%%%%%%%%%%%%%%%%%%%%%
%  Common mathematical sets.
%%%%%%%%%%%%%%%%%%%%%%%%%%%%%%%%%%%%%%%%%%%%%%%%%%%%%%%%

\newcommand{\A}{\mathbb{A}} % Algebraic Numbers
\newcommand{\C}{\mathbb{C}} % Complex
\newcommand{\F}{\mathbb{F}} % Field
\newcommand{\N}{\mathbb{N}} % Naturals
\newcommand{\Q}{\mathbb{Q}} % Rationals
\newcommand{\R}{\mathbb{R}} % Reals
\newcommand{\T}{\mathbb{T}} % Unimodular
\newcommand{\Z}{\mathbb{Z}} % Integers

%%%%%%%%%%%%%%%%%%%%%%%%%%%%%%%%%%%%%%%%%%%%%%%%%%%%%%%%
%  Fancy letters
%     (Note that A and S are missing --
%            \AA and \SS are something else)
%%%%%%%%%%%%%%%%%%%%%%%%%%%%%%%%%%%%%%%%%%%%%%%%%%%%%%%%
\newcommand{\BB}{\mathcal{B}}
\newcommand{\CE}{\mathcal{C}}
\newcommand{\DD}{\mathcal{D}}
\newcommand{\EE}{\mathcal{E}}
\newcommand{\FF}{\mathcal{F}}
\newcommand{\GG}{\mathcal{G}}
\newcommand{\HH}{\mathcal{H}}
\newcommand{\II}{\mathcal{I}}
\newcommand{\JJ}{\mathcal{J}}
\newcommand{\KK}{\mathcal{K}}
\newcommand{\LL}{\mathcal{L}}
\newcommand{\MM}{\mathcal{M}}
\newcommand{\NN}{\mathcal{N}}
\newcommand{\OO}{\mathcal{O}}
\newcommand{\PP}{\mathcal{P}}
\newcommand{\QQ}{\mathcal{Q}}
\newcommand{\RR}{\mathcal{R}}
\newcommand{\TT}{\mathcal{T}}
\newcommand{\UU}{\mathcal{U}}
\newcommand{\VV}{\mathcal{V}}
\newcommand{\WW}{\mathcal{W}}
\newcommand{\XX}{\mathcal{X}}
\newcommand{\YY}{\mathcal{Y}}
\newcommand{\ZZ}{\mathcal{Z}}

%%%%%%%%%%%%%%%%%%%%%%%%%%%%%%%%%%%%%%%%%%%%%%%%%%%%%%%%
%  Matrix entries -- Makes each entry uni-spaced.
%        Use only one column.
%   Example:
%   $$
%    W_7=\left(\begin{array}{c}
%          \Zp\Zp\Zp\Zp\Zz\Zz\Zz \\
%          \Zp\Zm\Zz\Zz\Zp\Zp\Zz \\
%          \Zp\Zz\Zm\Zz\Zm\Zz\Zp \\
%          \Zp\Zz\Zz\Zm\Zz\Zm\Zm \\
%          \Zz\Zp\Zm\Zz\Zz\Zp\Zm \\
%          \Zz\Zp\Zz\Zm\Zp\Zz\Zp \\
%          \Zz\Zz\Zp\Zm\Zm\Zp\Zz
%        \end{array}\right)
%   $$
%%%%%%%%%%%%%%%%%%%%%%%%%%%%%%%%%%%%%%%%%%%%%%%%%%%%%%%%
\newcommand\MatEntry[1]{\makebox[1.15em]{#1}}
\newcommand\Zp{\MatEntry{$1$}}
\newcommand\Zm{\MatEntry{$-$}}
\newcommand\Zz{\MatEntry{$0$}}
\newcommand\Zi{\MatEntry{$i$}}
\newcommand\Zj{\MatEntry{$j$}}
\newcommand\Ze{\MatEntry{$~$}} % Empty

\newcommand\Za{\MatEntry{$\omega$}}
\newcommand\ZA{\MatEntry{$\overline{\omega}$}}
\newcommand\Zb{\MatEntry{$\underline{\overline{\omega}}$}}
\newcommand\ZB{\MatEntry{$\underline{\omega}$}}

%%%%%%%%%%%%%%%%%%%%%%%%%%%%%%%%%%%%%%%%%%%%%%%%%%%%%%%%
%  Easy way to make vectors...
%%%%%%%%%%%%%%%%%%%%%%%%%%%%%%%%%%%%%%%%%%%%%%%%%%%%%%%%
\newcommand{\vecfive}[5]{($#1$ & $#2$ & $#3$ & $#4$ & $#5$)}
\newcommand{\vecseven}[7]{$\left(\begin{array}{ccccccc}#1 & #2 & #3 & #4 & #5 & #6 & #7\end{array}\right)$}
\newcommand{\veceight}[8]{$\left(\begin{array}{cccccccc}#1 & #2 & #3 & #4 & #5 & #6 & #7 & #8\end{array}\right)$}

%%%%%%%%%%%%%%%%%%%%%%%%%%%%%%%%%%%%%%%%%%%%%%%%%%%%%%%%
%  A different look for \begin{list}
%%%%%%%%%%%%%%%%%%%%%%%%%%%%%%%%%%%%%%%%%%%%%%%%%%%%%%%%
\newenvironment{myind}[1]%
 {\begin{list}{}%
    {\setlength{\leftmargin}{#1}}%
  \item[]%
 }
{\end{list}}

% \newcommand{\change}[2]{{\color{red} #1}~{\color{blue} #2}}
% \newcommand{\add}[1]{\change{}{#1}}
% \newcommand{\remove}[1]{\change{#1}{}}
